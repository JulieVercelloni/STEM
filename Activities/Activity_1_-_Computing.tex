\documentclass[]{article}
\usepackage{lmodern}
\usepackage{amssymb,amsmath}
\usepackage{ifxetex,ifluatex}
\usepackage{fixltx2e} % provides \textsubscript
\ifnum 0\ifxetex 1\fi\ifluatex 1\fi=0 % if pdftex
  \usepackage[T1]{fontenc}
  \usepackage[utf8]{inputenc}
\else % if luatex or xelatex
  \ifxetex
    \usepackage{mathspec}
  \else
    \usepackage{fontspec}
  \fi
  \defaultfontfeatures{Ligatures=TeX,Scale=MatchLowercase}
\fi
% use upquote if available, for straight quotes in verbatim environments
\IfFileExists{upquote.sty}{\usepackage{upquote}}{}
% use microtype if available
\IfFileExists{microtype.sty}{%
\usepackage{microtype}
\UseMicrotypeSet[protrusion]{basicmath} % disable protrusion for tt fonts
}{}
\usepackage[margin=1in]{geometry}
\usepackage{hyperref}
\hypersetup{unicode=true,
            pdftitle={Activity 1 - Data extraction},
            pdfauthor={Julie Vercelloni},
            pdfborder={0 0 0},
            breaklinks=true}
\urlstyle{same}  % don't use monospace font for urls
\usepackage{color}
\usepackage{fancyvrb}
\newcommand{\VerbBar}{|}
\newcommand{\VERB}{\Verb[commandchars=\\\{\}]}
\DefineVerbatimEnvironment{Highlighting}{Verbatim}{commandchars=\\\{\}}
% Add ',fontsize=\small' for more characters per line
\usepackage{framed}
\definecolor{shadecolor}{RGB}{248,248,248}
\newenvironment{Shaded}{\begin{snugshade}}{\end{snugshade}}
\newcommand{\KeywordTok}[1]{\textcolor[rgb]{0.13,0.29,0.53}{\textbf{#1}}}
\newcommand{\DataTypeTok}[1]{\textcolor[rgb]{0.13,0.29,0.53}{#1}}
\newcommand{\DecValTok}[1]{\textcolor[rgb]{0.00,0.00,0.81}{#1}}
\newcommand{\BaseNTok}[1]{\textcolor[rgb]{0.00,0.00,0.81}{#1}}
\newcommand{\FloatTok}[1]{\textcolor[rgb]{0.00,0.00,0.81}{#1}}
\newcommand{\ConstantTok}[1]{\textcolor[rgb]{0.00,0.00,0.00}{#1}}
\newcommand{\CharTok}[1]{\textcolor[rgb]{0.31,0.60,0.02}{#1}}
\newcommand{\SpecialCharTok}[1]{\textcolor[rgb]{0.00,0.00,0.00}{#1}}
\newcommand{\StringTok}[1]{\textcolor[rgb]{0.31,0.60,0.02}{#1}}
\newcommand{\VerbatimStringTok}[1]{\textcolor[rgb]{0.31,0.60,0.02}{#1}}
\newcommand{\SpecialStringTok}[1]{\textcolor[rgb]{0.31,0.60,0.02}{#1}}
\newcommand{\ImportTok}[1]{#1}
\newcommand{\CommentTok}[1]{\textcolor[rgb]{0.56,0.35,0.01}{\textit{#1}}}
\newcommand{\DocumentationTok}[1]{\textcolor[rgb]{0.56,0.35,0.01}{\textbf{\textit{#1}}}}
\newcommand{\AnnotationTok}[1]{\textcolor[rgb]{0.56,0.35,0.01}{\textbf{\textit{#1}}}}
\newcommand{\CommentVarTok}[1]{\textcolor[rgb]{0.56,0.35,0.01}{\textbf{\textit{#1}}}}
\newcommand{\OtherTok}[1]{\textcolor[rgb]{0.56,0.35,0.01}{#1}}
\newcommand{\FunctionTok}[1]{\textcolor[rgb]{0.00,0.00,0.00}{#1}}
\newcommand{\VariableTok}[1]{\textcolor[rgb]{0.00,0.00,0.00}{#1}}
\newcommand{\ControlFlowTok}[1]{\textcolor[rgb]{0.13,0.29,0.53}{\textbf{#1}}}
\newcommand{\OperatorTok}[1]{\textcolor[rgb]{0.81,0.36,0.00}{\textbf{#1}}}
\newcommand{\BuiltInTok}[1]{#1}
\newcommand{\ExtensionTok}[1]{#1}
\newcommand{\PreprocessorTok}[1]{\textcolor[rgb]{0.56,0.35,0.01}{\textit{#1}}}
\newcommand{\AttributeTok}[1]{\textcolor[rgb]{0.77,0.63,0.00}{#1}}
\newcommand{\RegionMarkerTok}[1]{#1}
\newcommand{\InformationTok}[1]{\textcolor[rgb]{0.56,0.35,0.01}{\textbf{\textit{#1}}}}
\newcommand{\WarningTok}[1]{\textcolor[rgb]{0.56,0.35,0.01}{\textbf{\textit{#1}}}}
\newcommand{\AlertTok}[1]{\textcolor[rgb]{0.94,0.16,0.16}{#1}}
\newcommand{\ErrorTok}[1]{\textcolor[rgb]{0.64,0.00,0.00}{\textbf{#1}}}
\newcommand{\NormalTok}[1]{#1}
\usepackage{graphicx,grffile}
\makeatletter
\def\maxwidth{\ifdim\Gin@nat@width>\linewidth\linewidth\else\Gin@nat@width\fi}
\def\maxheight{\ifdim\Gin@nat@height>\textheight\textheight\else\Gin@nat@height\fi}
\makeatother
% Scale images if necessary, so that they will not overflow the page
% margins by default, and it is still possible to overwrite the defaults
% using explicit options in \includegraphics[width, height, ...]{}
\setkeys{Gin}{width=\maxwidth,height=\maxheight,keepaspectratio}
\IfFileExists{parskip.sty}{%
\usepackage{parskip}
}{% else
\setlength{\parindent}{0pt}
\setlength{\parskip}{6pt plus 2pt minus 1pt}
}
\setlength{\emergencystretch}{3em}  % prevent overfull lines
\providecommand{\tightlist}{%
  \setlength{\itemsep}{0pt}\setlength{\parskip}{0pt}}
\setcounter{secnumdepth}{0}
% Redefines (sub)paragraphs to behave more like sections
\ifx\paragraph\undefined\else
\let\oldparagraph\paragraph
\renewcommand{\paragraph}[1]{\oldparagraph{#1}\mbox{}}
\fi
\ifx\subparagraph\undefined\else
\let\oldsubparagraph\subparagraph
\renewcommand{\subparagraph}[1]{\oldsubparagraph{#1}\mbox{}}
\fi

%%% Use protect on footnotes to avoid problems with footnotes in titles
\let\rmarkdownfootnote\footnote%
\def\footnote{\protect\rmarkdownfootnote}

%%% Change title format to be more compact
\usepackage{titling}

% Create subtitle command for use in maketitle
\newcommand{\subtitle}[1]{
  \posttitle{
    \begin{center}\large#1\end{center}
    }
}

\setlength{\droptitle}{-2em}

  \title{Activity 1 - Data extraction}
    \pretitle{\vspace{\droptitle}\centering\huge}
  \posttitle{\par}
    \author{Julie Vercelloni}
    \preauthor{\centering\large\emph}
  \postauthor{\par}
      \predate{\centering\large\emph}
  \postdate{\par}
    \date{28 March 2019}


\begin{document}
\maketitle

The following steps extract your own results from the online coral
classifications and corresponding responses from the experts. We are
going to manipulate three data tables in order to join them together.

\subsection{Step 1 - Load the R
libraries}\label{step-1---load-the-r-libraries}

\begin{Shaded}
\begin{Highlighting}[]
\KeywordTok{rm}\NormalTok{(}\DataTypeTok{list=}\KeywordTok{ls}\NormalTok{())}

\NormalTok{julie <-}\StringTok{ }\KeywordTok{Sys.info}\NormalTok{()[}\StringTok{"nodename"}\NormalTok{] }\OperatorTok{==}\StringTok{ "SEF-PA00130783"}

\ControlFlowTok{if}\NormalTok{ (julie)\{}
  \KeywordTok{.libPaths}\NormalTok{(}\StringTok{"C:}\CharTok{\textbackslash{}\textbackslash{}}\StringTok{Julie}\CharTok{\textbackslash{}\textbackslash{}}\StringTok{Rpkgs"}\NormalTok{)}
\NormalTok{\}}
\KeywordTok{library}\NormalTok{(dplyr)}
\end{Highlighting}
\end{Shaded}

\begin{verbatim}
## 
## Attaching package: 'dplyr'
\end{verbatim}

\begin{verbatim}
## The following objects are masked from 'package:stats':
## 
##     filter, lag
\end{verbatim}

\begin{verbatim}
## The following objects are masked from 'package:base':
## 
##     intersect, setdiff, setequal, union
\end{verbatim}

\begin{Shaded}
\begin{Highlighting}[]
\KeywordTok{library}\NormalTok{(tidyr)}
\end{Highlighting}
\end{Shaded}

\subsection{Step 2 - Creation of table
1}\label{step-2---creation-of-table-1}

The ``users\_save''" table contains information about your user id and
email address. The following code lines extract your user id (not known
yet) based on your email address and create table 1.

The ``id'' variable of table 1 is your joined field with table 2 (not
created yet).

\begin{Shaded}
\begin{Highlighting}[]
\KeywordTok{load}\NormalTok{(}\StringTok{"C:}\CharTok{\textbackslash{}\textbackslash{}}\StringTok{Users}\CharTok{\textbackslash{}\textbackslash{}}\StringTok{vercello}\CharTok{\textbackslash{}\textbackslash{}}\StringTok{Dropbox}\CharTok{\textbackslash{}\textbackslash{}}\StringTok{QUT}\CharTok{\textbackslash{}\textbackslash{}}\StringTok{STEM}\CharTok{\textbackslash{}\textbackslash{}}\StringTok{Data}\CharTok{\textbackslash{}\textbackslash{}}\StringTok{users_save.Rdata"}\NormalTok{)}
\NormalTok{your_email<-}\StringTok{"wussup111@gmail.com"}
\NormalTok{table1<-users_save}\OperatorTok\KeywordTok{filter}\NormalTok{(email}\OperatorTok{==}\NormalTok{your_email)}\OperatorTok\KeywordTok{data.frame}\NormalTok{()}
\end{Highlighting}
\end{Shaded}

\subsection{Step 3 - Creation of table
2}\label{step-3---creation-of-table-2}

The table 2 will contain all the results from the online classification
module and your associated userid and email address. A pre-manipulation
was made to extract the results from the students and saved them into
``elicitation\_save''.

\begin{Shaded}
\begin{Highlighting}[]
\KeywordTok{load}\NormalTok{(}\StringTok{"C:}\CharTok{\textbackslash{}\textbackslash{}}\StringTok{Users}\CharTok{\textbackslash{}\textbackslash{}}\StringTok{vercello}\CharTok{\textbackslash{}\textbackslash{}}\StringTok{Dropbox}\CharTok{\textbackslash{}\textbackslash{}}\StringTok{QUT}\CharTok{\textbackslash{}\textbackslash{}}\StringTok{STEM}\CharTok{\textbackslash{}\textbackslash{}}\StringTok{Data}\CharTok{\textbackslash{}\textbackslash{}}\StringTok{elicitation_save.Rdata"}\NormalTok{)}
\KeywordTok{names}\NormalTok{(eli_save)}
\end{Highlighting}
\end{Shaded}

\begin{verbatim}
## [1] "mediaID"          "id"               "Date"            
## [4] "xadj"             "yadj"             "Classification 2"
\end{verbatim}

\begin{Shaded}
\begin{Highlighting}[]
\NormalTok{my.eli<-eli_save}\OperatorTok\KeywordTok{filter}\NormalTok{(id}\OperatorTok\NormalTok{table1}\OperatorTok{$}\NormalTok{id)}
\end{Highlighting}
\end{Shaded}

``my.eli'' table canbe used to look at:

\begin{itemize}
\tightlist
\item
  number of images that you have classified:
\end{itemize}

\begin{Shaded}
\begin{Highlighting}[]
\KeywordTok{length}\NormalTok{(}\KeywordTok{unique}\NormalTok{(my.eli}\OperatorTok{$}\NormalTok{mediaID))}
\end{Highlighting}
\end{Shaded}

\begin{verbatim}
## [1] 216
\end{verbatim}

\begin{itemize}
\tightlist
\item
  number of classification per images:
\end{itemize}

\begin{Shaded}
\begin{Highlighting}[]
\NormalTok{my.eli}\OperatorTok\KeywordTok{group_by}\NormalTok{(mediaID)}\OperatorTok\KeywordTok{tally}\NormalTok{()}
\end{Highlighting}
\end{Shaded}

\begin{verbatim}
## # A tibble: 216 x 2
##        mediaID     n
##          <dbl> <int>
##  1 10001020702    15
##  2 10001020902    15
##  3 10001021702    15
##  4 10001023502    15
##  5 10001025502    15
##  6 10001025802    30
##  7 12001090002    15
##  8 12001090402    15
##  9 12001091202    30
## 10 12001091502    15
## # ... with 206 more rows
\end{verbatim}

\begin{itemize}
\tightlist
\item
  number of times you classified a hard corals (hard) per images
\end{itemize}

\begin{Shaded}
\begin{Highlighting}[]
\NormalTok{my.eli}\OperatorTok\KeywordTok{group_by}\NormalTok{(mediaID,}\StringTok{`}\DataTypeTok{Classification 2}\StringTok{`}\NormalTok{)}\OperatorTok\KeywordTok{summarise}\NormalTok{(}\DataTypeTok{Count=}\KeywordTok{n}\NormalTok{())}
\end{Highlighting}
\end{Shaded}

\begin{verbatim}
## # A tibble: 226 x 3
## # Groups:   mediaID [?]
##        mediaID `Classification 2` Count
##          <dbl> <fct>              <int>
##  1 10001020702 hard                  15
##  2 10001020902 hard                  15
##  3 10001021702 hard                  15
##  4 10001023502 hard                  15
##  5 10001025502 hard                  15
##  6 10001025802 hard                  30
##  7 12001090002 hard                  15
##  8 12001090402 hard                  15
##  9 12001091202 hard                  30
## 10 12001091502 hard                  15
## # ... with 216 more rows
\end{verbatim}

Now we want to join my.eli and table1 together and create a new column
named ``Annotator 2'' showing your email address

\begin{Shaded}
\begin{Highlighting}[]
\NormalTok{table2<-}\KeywordTok{inner_join}\NormalTok{(my.eli,table1)}
\end{Highlighting}
\end{Shaded}

\begin{verbatim}
## Joining, by = "id"
\end{verbatim}

\begin{Shaded}
\begin{Highlighting}[]
\KeywordTok{colnames}\NormalTok{(table2)<-}\KeywordTok{c}\NormalTok{(}\StringTok{"mediaID"}\NormalTok{,}\StringTok{"id"}\NormalTok{,}\StringTok{"Date"}\NormalTok{,}\StringTok{"xadj"}\NormalTok{,}\StringTok{"yadj"}\NormalTok{,}\StringTok{"Classification 2"}\NormalTok{,}\StringTok{"Annotator 2"}\NormalTok{)}
\end{Highlighting}
\end{Shaded}

Let's clean the factors that composed the variable ``Annotator 2'' for
the next step.

\begin{Shaded}
\begin{Highlighting}[]
\NormalTok{table2<-table2}\OperatorTok\KeywordTok{mutate}\NormalTok{(}\StringTok{`}\DataTypeTok{Classification 2}\StringTok{`}\NormalTok{=(}\KeywordTok{ifelse}\NormalTok{(}\StringTok{`}\DataTypeTok{Classification 2}\StringTok{`}\OperatorTok{==}\StringTok{"hard"}\NormalTok{,}\StringTok{"Hard Corals"}\NormalTok{,}
                                          \KeywordTok{ifelse}\NormalTok{(}\StringTok{`}\DataTypeTok{Classification 2}\StringTok{`}\OperatorTok{==}\StringTok{"algue"}\NormalTok{,}\StringTok{"Algae"}\NormalTok{,}
                                              \KeywordTok{ifelse}\NormalTok{(}\StringTok{`}\DataTypeTok{Classification 2}\StringTok{`}\OperatorTok{==}\StringTok{"soft"}\NormalTok{,}\StringTok{"Soft Corals"}\NormalTok{,}
                                                \KeywordTok{ifelse}\NormalTok{(}\StringTok{`}\DataTypeTok{Classification 2}\StringTok{`}\OperatorTok{==}\StringTok{"sand"}\NormalTok{,}\StringTok{"Sand"}\NormalTok{,}
                                                       \KeywordTok{ifelse}\NormalTok{(}\StringTok{`}\DataTypeTok{Classification 2}\StringTok{`}\OperatorTok{==}\StringTok{"other"}\NormalTok{,}\StringTok{"Other"}\NormalTok{,}\StringTok{"Not_benthic"}\NormalTok{)))))))}
\end{Highlighting}
\end{Shaded}

\subsection{Step 4 - Creation of table
3}\label{step-4---creation-of-table-3}

The last table contains the classification from the experts.

\begin{Shaded}
\begin{Highlighting}[]
\KeywordTok{load}\NormalTok{(}\StringTok{"C:}\CharTok{\textbackslash{}\textbackslash{}}\StringTok{Users}\CharTok{\textbackslash{}\textbackslash{}}\StringTok{vercello}\CharTok{\textbackslash{}\textbackslash{}}\StringTok{Dropbox}\CharTok{\textbackslash{}\textbackslash{}}\StringTok{QUT}\CharTok{\textbackslash{}\textbackslash{}}\StringTok{STEM}\CharTok{\textbackslash{}\textbackslash{}}\StringTok{Data}\CharTok{\textbackslash{}\textbackslash{}}\StringTok{expert_save.Rdata"}\NormalTok{)}

\NormalTok{table3<-exp_save}\OperatorTok\KeywordTok{mutate}\NormalTok{(}\DataTypeTok{Classification=}\NormalTok{(}\KeywordTok{ifelse}\NormalTok{(Classification}\OperatorTok{==}\StringTok{"Hard corals"}\NormalTok{,}\StringTok{"Hard Corals"}\NormalTok{,}
                                                \KeywordTok{ifelse}\NormalTok{(Classification}\OperatorTok{==}\StringTok{"Hard Corals"}\NormalTok{,}\StringTok{"Hard Corals"}\NormalTok{,}
                                                \KeywordTok{ifelse}\NormalTok{(Classification}\OperatorTok{==}\StringTok{"Algae"}\NormalTok{,}\StringTok{"Algae"}\NormalTok{,}
                                                       \KeywordTok{ifelse}\NormalTok{(Classification}\OperatorTok{==}\StringTok{"Soft Coral"}\NormalTok{,}\StringTok{"Soft Corals"}\NormalTok{,}
                                                              \KeywordTok{ifelse}\NormalTok{(Classification}\OperatorTok{==}\StringTok{"Soft Corals"}\NormalTok{,}\StringTok{"Soft Corals"}\NormalTok{,}
                                                              \KeywordTok{ifelse}\NormalTok{(Classification}\OperatorTok{==}\StringTok{"Sand"}\NormalTok{,}\StringTok{"Sand"}\NormalTok{,}
                                                                \KeywordTok{ifelse}\NormalTok{(Classification}\OperatorTok{==}\StringTok{"Other"}\NormalTok{,}\StringTok{"Other"}\NormalTok{,}\StringTok{"Not_benthic"}\NormalTok{)))))))))}

\KeywordTok{colnames}\NormalTok{(table3)<-}\KeywordTok{c}\NormalTok{(}\StringTok{"mediaID"}\NormalTok{,}\StringTok{"reef_name"}\NormalTok{,}\StringTok{"lng"}\NormalTok{,}\StringTok{"lat"}\NormalTok{,}\StringTok{"year"}\NormalTok{,}\StringTok{"Annotator1"}\NormalTok{,}\StringTok{"xadj"}\NormalTok{,}\StringTok{"yadj"}\NormalTok{,}\StringTok{"Classification 1"}\NormalTok{)}
\end{Highlighting}
\end{Shaded}

\subsection{Step 5 - Create and save the final
table}\label{step-5---create-and-save-the-final-table}

Similarly the previous step, the final table is built by joining table 2
and table 3 together.

\begin{Shaded}
\begin{Highlighting}[]
\NormalTok{table_final<-}\KeywordTok{inner_join}\NormalTok{(table2,table3)}
\end{Highlighting}
\end{Shaded}

\begin{verbatim}
## Joining, by = c("mediaID", "xadj", "yadj")
\end{verbatim}

\begin{Shaded}
\begin{Highlighting}[]
\KeywordTok{save}\NormalTok{(table_final,}\DataTypeTok{file=}\StringTok{"C:}\CharTok{\textbackslash{}\textbackslash{}}\StringTok{Users}\CharTok{\textbackslash{}\textbackslash{}}\StringTok{vercello}\CharTok{\textbackslash{}\textbackslash{}}\StringTok{Dropbox}\CharTok{\textbackslash{}\textbackslash{}}\StringTok{QUT}\CharTok{\textbackslash{}\textbackslash{}}\StringTok{STEM}\CharTok{\textbackslash{}\textbackslash{}}\StringTok{Data}\CharTok{\textbackslash{}\textbackslash{}}\StringTok{table_final.Rdata"}\NormalTok{)}
\end{Highlighting}
\end{Shaded}

\subsection{Step 6 - Questions}\label{step-6---questions}

Use table\_final to answer the following questions

\begin{enumerate}
\def\labelenumi{\arabic{enumi}.}
\tightlist
\item
  Have many unique coral reefs have you visited?\\
\item
  Why the variables ``xadj'' and ``yadj'' are important?
\item
  Extract the total number of time that the ``Annotator 1'' (expert)
  classified ``Hard Corals''
\end{enumerate}


\end{document}
