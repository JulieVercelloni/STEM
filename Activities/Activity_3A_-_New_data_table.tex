\documentclass[]{article}
\usepackage{lmodern}
\usepackage{amssymb,amsmath}
\usepackage{ifxetex,ifluatex}
\usepackage{fixltx2e} % provides \textsubscript
\ifnum 0\ifxetex 1\fi\ifluatex 1\fi=0 % if pdftex
  \usepackage[T1]{fontenc}
  \usepackage[utf8]{inputenc}
\else % if luatex or xelatex
  \ifxetex
    \usepackage{mathspec}
  \else
    \usepackage{fontspec}
  \fi
  \defaultfontfeatures{Ligatures=TeX,Scale=MatchLowercase}
\fi
% use upquote if available, for straight quotes in verbatim environments
\IfFileExists{upquote.sty}{\usepackage{upquote}}{}
% use microtype if available
\IfFileExists{microtype.sty}{%
\usepackage{microtype}
\UseMicrotypeSet[protrusion]{basicmath} % disable protrusion for tt fonts
}{}
\usepackage[margin=1in]{geometry}
\usepackage{hyperref}
\hypersetup{unicode=true,
            pdftitle={Activity 3A - Compute the group data table},
            pdfborder={0 0 0},
            breaklinks=true}
\urlstyle{same}  % don't use monospace font for urls
\usepackage{color}
\usepackage{fancyvrb}
\newcommand{\VerbBar}{|}
\newcommand{\VERB}{\Verb[commandchars=\\\{\}]}
\DefineVerbatimEnvironment{Highlighting}{Verbatim}{commandchars=\\\{\}}
% Add ',fontsize=\small' for more characters per line
\usepackage{framed}
\definecolor{shadecolor}{RGB}{248,248,248}
\newenvironment{Shaded}{\begin{snugshade}}{\end{snugshade}}
\newcommand{\KeywordTok}[1]{\textcolor[rgb]{0.13,0.29,0.53}{\textbf{#1}}}
\newcommand{\DataTypeTok}[1]{\textcolor[rgb]{0.13,0.29,0.53}{#1}}
\newcommand{\DecValTok}[1]{\textcolor[rgb]{0.00,0.00,0.81}{#1}}
\newcommand{\BaseNTok}[1]{\textcolor[rgb]{0.00,0.00,0.81}{#1}}
\newcommand{\FloatTok}[1]{\textcolor[rgb]{0.00,0.00,0.81}{#1}}
\newcommand{\ConstantTok}[1]{\textcolor[rgb]{0.00,0.00,0.00}{#1}}
\newcommand{\CharTok}[1]{\textcolor[rgb]{0.31,0.60,0.02}{#1}}
\newcommand{\SpecialCharTok}[1]{\textcolor[rgb]{0.00,0.00,0.00}{#1}}
\newcommand{\StringTok}[1]{\textcolor[rgb]{0.31,0.60,0.02}{#1}}
\newcommand{\VerbatimStringTok}[1]{\textcolor[rgb]{0.31,0.60,0.02}{#1}}
\newcommand{\SpecialStringTok}[1]{\textcolor[rgb]{0.31,0.60,0.02}{#1}}
\newcommand{\ImportTok}[1]{#1}
\newcommand{\CommentTok}[1]{\textcolor[rgb]{0.56,0.35,0.01}{\textit{#1}}}
\newcommand{\DocumentationTok}[1]{\textcolor[rgb]{0.56,0.35,0.01}{\textbf{\textit{#1}}}}
\newcommand{\AnnotationTok}[1]{\textcolor[rgb]{0.56,0.35,0.01}{\textbf{\textit{#1}}}}
\newcommand{\CommentVarTok}[1]{\textcolor[rgb]{0.56,0.35,0.01}{\textbf{\textit{#1}}}}
\newcommand{\OtherTok}[1]{\textcolor[rgb]{0.56,0.35,0.01}{#1}}
\newcommand{\FunctionTok}[1]{\textcolor[rgb]{0.00,0.00,0.00}{#1}}
\newcommand{\VariableTok}[1]{\textcolor[rgb]{0.00,0.00,0.00}{#1}}
\newcommand{\ControlFlowTok}[1]{\textcolor[rgb]{0.13,0.29,0.53}{\textbf{#1}}}
\newcommand{\OperatorTok}[1]{\textcolor[rgb]{0.81,0.36,0.00}{\textbf{#1}}}
\newcommand{\BuiltInTok}[1]{#1}
\newcommand{\ExtensionTok}[1]{#1}
\newcommand{\PreprocessorTok}[1]{\textcolor[rgb]{0.56,0.35,0.01}{\textit{#1}}}
\newcommand{\AttributeTok}[1]{\textcolor[rgb]{0.77,0.63,0.00}{#1}}
\newcommand{\RegionMarkerTok}[1]{#1}
\newcommand{\InformationTok}[1]{\textcolor[rgb]{0.56,0.35,0.01}{\textbf{\textit{#1}}}}
\newcommand{\WarningTok}[1]{\textcolor[rgb]{0.56,0.35,0.01}{\textbf{\textit{#1}}}}
\newcommand{\AlertTok}[1]{\textcolor[rgb]{0.94,0.16,0.16}{#1}}
\newcommand{\ErrorTok}[1]{\textcolor[rgb]{0.64,0.00,0.00}{\textbf{#1}}}
\newcommand{\NormalTok}[1]{#1}
\usepackage{graphicx,grffile}
\makeatletter
\def\maxwidth{\ifdim\Gin@nat@width>\linewidth\linewidth\else\Gin@nat@width\fi}
\def\maxheight{\ifdim\Gin@nat@height>\textheight\textheight\else\Gin@nat@height\fi}
\makeatother
% Scale images if necessary, so that they will not overflow the page
% margins by default, and it is still possible to overwrite the defaults
% using explicit options in \includegraphics[width, height, ...]{}
\setkeys{Gin}{width=\maxwidth,height=\maxheight,keepaspectratio}
\IfFileExists{parskip.sty}{%
\usepackage{parskip}
}{% else
\setlength{\parindent}{0pt}
\setlength{\parskip}{6pt plus 2pt minus 1pt}
}
\setlength{\emergencystretch}{3em}  % prevent overfull lines
\providecommand{\tightlist}{%
  \setlength{\itemsep}{0pt}\setlength{\parskip}{0pt}}
\setcounter{secnumdepth}{0}
% Redefines (sub)paragraphs to behave more like sections
\ifx\paragraph\undefined\else
\let\oldparagraph\paragraph
\renewcommand{\paragraph}[1]{\oldparagraph{#1}\mbox{}}
\fi
\ifx\subparagraph\undefined\else
\let\oldsubparagraph\subparagraph
\renewcommand{\subparagraph}[1]{\oldsubparagraph{#1}\mbox{}}
\fi

%%% Use protect on footnotes to avoid problems with footnotes in titles
\let\rmarkdownfootnote\footnote%
\def\footnote{\protect\rmarkdownfootnote}

%%% Change title format to be more compact
\usepackage{titling}

% Create subtitle command for use in maketitle
\newcommand{\subtitle}[1]{
  \posttitle{
    \begin{center}\large#1\end{center}
    }
}

\setlength{\droptitle}{-2em}

  \title{Activity 3A - Compute the group data table}
    \pretitle{\vspace{\droptitle}\centering\huge}
  \posttitle{\par}
    \author{true}
    \preauthor{\centering\large\emph}
  \postauthor{\par}
      \predate{\centering\large\emph}
  \postdate{\par}
    \date{08 April, 2019}


\begin{document}
\maketitle

\begin{center}\rule{0.5\linewidth}{\linethickness}\end{center}

The following code will create a new data table for each group

\subsection{Loading the R Libraries}\label{loading-the-r-libraries}

\begin{Shaded}
\begin{Highlighting}[]
\KeywordTok{rm}\NormalTok{(}\DataTypeTok{list=}\KeywordTok{ls}\NormalTok{())}
\KeywordTok{library}\NormalTok{(dplyr)}
\KeywordTok{library}\NormalTok{(tidyr)}
\end{Highlighting}
\end{Shaded}

\subsection{Creating Table 1}\label{creating-table-1}

The \texttt{users\_save.Rdata} table contains information about your
user id and email address. The following code lines extract your user
id's (not known yet) based on your email addresses and creates
\texttt{table1}.

The \texttt{id} variable of \texttt{table1} is the field used to join it
with \texttt{table2} (not created yet).

\begin{Shaded}
\begin{Highlighting}[]
\KeywordTok{load}\NormalTok{(}\StringTok{"../Data/users_save.Rdata"}\NormalTok{)}

\NormalTok{group_emails <-}\StringTok{ }\KeywordTok{c}\NormalTok{(}\StringTok{"julie.vercelloni@gmail.com"}\NormalTok{,}\StringTok{"erin_test@qut.edu.au"}\NormalTok{,}\StringTok{"k.mengersen@qut.edu.au"}\NormalTok{) }\CommentTok{# change and add email of your group members }
\NormalTok{table1 <-}\StringTok{ }\NormalTok{users_save }\OperatorTok
\StringTok{  }\KeywordTok{filter}\NormalTok{(email }\OperatorTok\StringTok{ }\NormalTok{group_emails) }\OperatorTok
\StringTok{  }\KeywordTok{data.frame}\NormalTok{()}
\end{Highlighting}
\end{Shaded}

\subsection{Creating Table 2}\label{creating-table-2}

\texttt{table2} will contain all the results from the online
classification module, your user id's, and email addresses. The results
from the students were already extracted, manipulated, and saved to
\texttt{elicitation\_save.Rdata}.

\begin{Shaded}
\begin{Highlighting}[]
\KeywordTok{load}\NormalTok{(}\StringTok{"../Data/elicitation_save.Rdata"}\NormalTok{)}
\KeywordTok{names}\NormalTok{(eli_save)}
\end{Highlighting}
\end{Shaded}

\begin{verbatim}
## [1] "mediaID"          "id"               "Date"            
## [4] "xadj"             "yadj"             "Classification 2"
\end{verbatim}

\begin{Shaded}
\begin{Highlighting}[]
\NormalTok{my.eli <-}\StringTok{ }\NormalTok{eli_save }\OperatorTok
\StringTok{  }\KeywordTok{filter}\NormalTok{(id }\OperatorTok\StringTok{ }\NormalTok{table1}\OperatorTok{$}\NormalTok{id)}
\end{Highlighting}
\end{Shaded}

\subsubsection{Joining the Data}\label{joining-the-data}

Now we want to join \texttt{my.eli} and \texttt{table1}. The table
created will have a new column called \texttt{Annotator\ 2} that shows
your email address.

\begin{Shaded}
\begin{Highlighting}[]
\NormalTok{table2 <-}\StringTok{ }\KeywordTok{inner_join}\NormalTok{(my.eli, table1)}
\KeywordTok{colnames}\NormalTok{(table2)<-}\KeywordTok{c}\NormalTok{(}\StringTok{"mediaID"}\NormalTok{,}
                    \StringTok{"id"}\NormalTok{,}
                    \StringTok{"Date"}\NormalTok{,}
                    \StringTok{"xadj"}\NormalTok{,}
                    \StringTok{"yadj"}\NormalTok{,}
                    \StringTok{"Classification 2"}\NormalTok{,}
                    \StringTok{"Annotator 2"}\NormalTok{)}
\end{Highlighting}
\end{Shaded}

Then we need to clean up the factors of this new variable
\texttt{Annotator\ 2} for the next step.

\begin{Shaded}
\begin{Highlighting}[]
\NormalTok{table2 <-}\StringTok{ }\NormalTok{table2 }\OperatorTok\StringTok{ }
\StringTok{  }\KeywordTok{mutate}\NormalTok{(}\StringTok{`}\DataTypeTok{Classification 2}\StringTok{`}\NormalTok{ =}\StringTok{ }\KeywordTok{case_when}\NormalTok{(}\StringTok{`}\DataTypeTok{Classification 2}\StringTok{`} \OperatorTok{==}\StringTok{ "hard"} \OperatorTok{~}\StringTok{ "Hard Corals"}\NormalTok{,}
                                        \StringTok{`}\DataTypeTok{Classification 2}\StringTok{`} \OperatorTok{==}\StringTok{ "algue"} \OperatorTok{~}\StringTok{ "Algae"}\NormalTok{,}
                                        \StringTok{`}\DataTypeTok{Classification 2}\StringTok{`} \OperatorTok{==}\StringTok{"soft"} \OperatorTok{~}\StringTok{ "Soft Corals"}\NormalTok{,}
                                        \StringTok{`}\DataTypeTok{Classification 2}\StringTok{`} \OperatorTok{==}\StringTok{ "sand"} \OperatorTok{~}\StringTok{ "Sand"}\NormalTok{,}
                                        \StringTok{`}\DataTypeTok{Classification 2}\StringTok{`} \OperatorTok{==}\StringTok{ "other"} \OperatorTok{~}\StringTok{ "Other"}\NormalTok{,}
                                        \OtherTok{TRUE} \OperatorTok{~}\StringTok{ "Not Benthic"}\NormalTok{))}
\end{Highlighting}
\end{Shaded}

\subsection{Creating Table 3}\label{creating-table-3}

The last table contains the classification from the experts.

\begin{Shaded}
\begin{Highlighting}[]
\KeywordTok{load}\NormalTok{(}\StringTok{"../Data/expert_save.Rdata"}\NormalTok{)}

\NormalTok{table3 <-}\StringTok{ }\NormalTok{exp_save }\OperatorTok\StringTok{ }
\StringTok{  }\KeywordTok{mutate}\NormalTok{(}\DataTypeTok{Classification =} \KeywordTok{case_when}\NormalTok{(Classification }\OperatorTok{==}\StringTok{ "Hard corals"} \OperatorTok{~}\StringTok{ "Hard Corals"}\NormalTok{,}
\NormalTok{                                    Classification }\OperatorTok{==}\StringTok{ "Hard Corals"} \OperatorTok{~}\StringTok{ "Hard Corals"}\NormalTok{,}
\NormalTok{                                    Classification }\OperatorTok{==}\StringTok{ "Algae"} \OperatorTok{~}\StringTok{ "Algae"}\NormalTok{,}
\NormalTok{                                    Classification }\OperatorTok{==}\StringTok{ "Soft Coral"} \OperatorTok{~}\StringTok{ "Soft Corals"}\NormalTok{,}
\NormalTok{                                    Classification }\OperatorTok{==}\StringTok{ "Soft Corals"} \OperatorTok{~}\StringTok{ "Soft Corals"}\NormalTok{,}
\NormalTok{                                    Classification }\OperatorTok{==}\StringTok{ "Sand"} \OperatorTok{~}\StringTok{ "Sand"}\NormalTok{,}
\NormalTok{                                    Classification }\OperatorTok{==}\StringTok{ "Other"} \OperatorTok{~}\StringTok{ "Other"}\NormalTok{,}
                                    \OtherTok{TRUE} \OperatorTok{~}\StringTok{ "Not Benthic"}\NormalTok{))}

\KeywordTok{colnames}\NormalTok{(table3) <-}\StringTok{ }\KeywordTok{c}\NormalTok{(}\StringTok{"mediaID"}\NormalTok{,}
                      \StringTok{"reef_name"}\NormalTok{,}
                      \StringTok{"lng"}\NormalTok{,}
                      \StringTok{"lat"}\NormalTok{,}
                      \StringTok{"year"}\NormalTok{,}
                      \StringTok{"Annotator1"}\NormalTok{,}
                      \StringTok{"xadj"}\NormalTok{,}
                      \StringTok{"yadj"}\NormalTok{,}
                      \StringTok{"Classification 1"}\NormalTok{)}
\end{Highlighting}
\end{Shaded}

\subsection{Create and Save the Final
Table}\label{create-and-save-the-final-table}

The final table \texttt{table\_final} is now built by joining
\texttt{table2} and \texttt{table3} together.

\begin{Shaded}
\begin{Highlighting}[]
\NormalTok{table_final <-}\StringTok{ }\KeywordTok{inner_join}\NormalTok{(table2, table3)}
\KeywordTok{save}\NormalTok{(table_final, }\DataTypeTok{file =} \StringTok{"../Data/table_final_group.Rdata"}\NormalTok{)}
\end{Highlighting}
\end{Shaded}


\end{document}
