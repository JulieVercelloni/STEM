\documentclass[]{article}
\usepackage{lmodern}
\usepackage{amssymb,amsmath}
\usepackage{ifxetex,ifluatex}
\usepackage{fixltx2e} % provides \textsubscript
\ifnum 0\ifxetex 1\fi\ifluatex 1\fi=0 % if pdftex
  \usepackage[T1]{fontenc}
  \usepackage[utf8]{inputenc}
\else % if luatex or xelatex
  \ifxetex
    \usepackage{mathspec}
  \else
    \usepackage{fontspec}
  \fi
  \defaultfontfeatures{Ligatures=TeX,Scale=MatchLowercase}
\fi
% use upquote if available, for straight quotes in verbatim environments
\IfFileExists{upquote.sty}{\usepackage{upquote}}{}
% use microtype if available
\IfFileExists{microtype.sty}{%
\usepackage{microtype}
\UseMicrotypeSet[protrusion]{basicmath} % disable protrusion for tt fonts
}{}
\usepackage[margin=1in]{geometry}
\usepackage{hyperref}
\hypersetup{unicode=true,
            pdftitle={Activity 2 - Accuracy estimations},
            pdfauthor={Julie Vercelloni},
            pdfborder={0 0 0},
            breaklinks=true}
\urlstyle{same}  % don't use monospace font for urls
\usepackage{color}
\usepackage{fancyvrb}
\newcommand{\VerbBar}{|}
\newcommand{\VERB}{\Verb[commandchars=\\\{\}]}
\DefineVerbatimEnvironment{Highlighting}{Verbatim}{commandchars=\\\{\}}
% Add ',fontsize=\small' for more characters per line
\usepackage{framed}
\definecolor{shadecolor}{RGB}{248,248,248}
\newenvironment{Shaded}{\begin{snugshade}}{\end{snugshade}}
\newcommand{\KeywordTok}[1]{\textcolor[rgb]{0.13,0.29,0.53}{\textbf{#1}}}
\newcommand{\DataTypeTok}[1]{\textcolor[rgb]{0.13,0.29,0.53}{#1}}
\newcommand{\DecValTok}[1]{\textcolor[rgb]{0.00,0.00,0.81}{#1}}
\newcommand{\BaseNTok}[1]{\textcolor[rgb]{0.00,0.00,0.81}{#1}}
\newcommand{\FloatTok}[1]{\textcolor[rgb]{0.00,0.00,0.81}{#1}}
\newcommand{\ConstantTok}[1]{\textcolor[rgb]{0.00,0.00,0.00}{#1}}
\newcommand{\CharTok}[1]{\textcolor[rgb]{0.31,0.60,0.02}{#1}}
\newcommand{\SpecialCharTok}[1]{\textcolor[rgb]{0.00,0.00,0.00}{#1}}
\newcommand{\StringTok}[1]{\textcolor[rgb]{0.31,0.60,0.02}{#1}}
\newcommand{\VerbatimStringTok}[1]{\textcolor[rgb]{0.31,0.60,0.02}{#1}}
\newcommand{\SpecialStringTok}[1]{\textcolor[rgb]{0.31,0.60,0.02}{#1}}
\newcommand{\ImportTok}[1]{#1}
\newcommand{\CommentTok}[1]{\textcolor[rgb]{0.56,0.35,0.01}{\textit{#1}}}
\newcommand{\DocumentationTok}[1]{\textcolor[rgb]{0.56,0.35,0.01}{\textbf{\textit{#1}}}}
\newcommand{\AnnotationTok}[1]{\textcolor[rgb]{0.56,0.35,0.01}{\textbf{\textit{#1}}}}
\newcommand{\CommentVarTok}[1]{\textcolor[rgb]{0.56,0.35,0.01}{\textbf{\textit{#1}}}}
\newcommand{\OtherTok}[1]{\textcolor[rgb]{0.56,0.35,0.01}{#1}}
\newcommand{\FunctionTok}[1]{\textcolor[rgb]{0.00,0.00,0.00}{#1}}
\newcommand{\VariableTok}[1]{\textcolor[rgb]{0.00,0.00,0.00}{#1}}
\newcommand{\ControlFlowTok}[1]{\textcolor[rgb]{0.13,0.29,0.53}{\textbf{#1}}}
\newcommand{\OperatorTok}[1]{\textcolor[rgb]{0.81,0.36,0.00}{\textbf{#1}}}
\newcommand{\BuiltInTok}[1]{#1}
\newcommand{\ExtensionTok}[1]{#1}
\newcommand{\PreprocessorTok}[1]{\textcolor[rgb]{0.56,0.35,0.01}{\textit{#1}}}
\newcommand{\AttributeTok}[1]{\textcolor[rgb]{0.77,0.63,0.00}{#1}}
\newcommand{\RegionMarkerTok}[1]{#1}
\newcommand{\InformationTok}[1]{\textcolor[rgb]{0.56,0.35,0.01}{\textbf{\textit{#1}}}}
\newcommand{\WarningTok}[1]{\textcolor[rgb]{0.56,0.35,0.01}{\textbf{\textit{#1}}}}
\newcommand{\AlertTok}[1]{\textcolor[rgb]{0.94,0.16,0.16}{#1}}
\newcommand{\ErrorTok}[1]{\textcolor[rgb]{0.64,0.00,0.00}{\textbf{#1}}}
\newcommand{\NormalTok}[1]{#1}
\usepackage{graphicx,grffile}
\makeatletter
\def\maxwidth{\ifdim\Gin@nat@width>\linewidth\linewidth\else\Gin@nat@width\fi}
\def\maxheight{\ifdim\Gin@nat@height>\textheight\textheight\else\Gin@nat@height\fi}
\makeatother
% Scale images if necessary, so that they will not overflow the page
% margins by default, and it is still possible to overwrite the defaults
% using explicit options in \includegraphics[width, height, ...]{}
\setkeys{Gin}{width=\maxwidth,height=\maxheight,keepaspectratio}
\IfFileExists{parskip.sty}{%
\usepackage{parskip}
}{% else
\setlength{\parindent}{0pt}
\setlength{\parskip}{6pt plus 2pt minus 1pt}
}
\setlength{\emergencystretch}{3em}  % prevent overfull lines
\providecommand{\tightlist}{%
  \setlength{\itemsep}{0pt}\setlength{\parskip}{0pt}}
\setcounter{secnumdepth}{0}
% Redefines (sub)paragraphs to behave more like sections
\ifx\paragraph\undefined\else
\let\oldparagraph\paragraph
\renewcommand{\paragraph}[1]{\oldparagraph{#1}\mbox{}}
\fi
\ifx\subparagraph\undefined\else
\let\oldsubparagraph\subparagraph
\renewcommand{\subparagraph}[1]{\oldsubparagraph{#1}\mbox{}}
\fi

%%% Use protect on footnotes to avoid problems with footnotes in titles
\let\rmarkdownfootnote\footnote%
\def\footnote{\protect\rmarkdownfootnote}

%%% Change title format to be more compact
\usepackage{titling}

% Create subtitle command for use in maketitle
\newcommand{\subtitle}[1]{
  \posttitle{
    \begin{center}\large#1\end{center}
    }
}

\setlength{\droptitle}{-2em}

  \title{Activity 2 - Accuracy estimations}
    \pretitle{\vspace{\droptitle}\centering\huge}
  \posttitle{\par}
    \author{Julie Vercelloni}
    \preauthor{\centering\large\emph}
  \postauthor{\par}
      \predate{\centering\large\emph}
  \postdate{\par}
    \date{29 March 2019}


\begin{document}
\maketitle

Using the table\_final from Activity 1, we are now going to estimate
your accuracy for reef image classifications.

\subsection{Load the R libraries}\label{load-the-r-libraries}

\begin{Shaded}
\begin{Highlighting}[]
\KeywordTok{rm}\NormalTok{(}\DataTypeTok{list=}\KeywordTok{ls}\NormalTok{())}

\NormalTok{julie <-}\StringTok{ }\KeywordTok{Sys.info}\NormalTok{()[}\StringTok{"nodename"}\NormalTok{] }\OperatorTok{==}\StringTok{ "SEF-PA00130783"}

\ControlFlowTok{if}\NormalTok{ (julie)\{}
  \KeywordTok{.libPaths}\NormalTok{(}\StringTok{"C:}\CharTok{\textbackslash{}\textbackslash{}}\StringTok{Julie}\CharTok{\textbackslash{}\textbackslash{}}\StringTok{Rpkgs"}\NormalTok{)}
\NormalTok{\}}
\KeywordTok{library}\NormalTok{(dplyr)}
\end{Highlighting}
\end{Shaded}

\begin{verbatim}
## 
## Attaching package: 'dplyr'
\end{verbatim}

\begin{verbatim}
## The following objects are masked from 'package:stats':
## 
##     filter, lag
\end{verbatim}

\begin{verbatim}
## The following objects are masked from 'package:base':
## 
##     intersect, setdiff, setequal, union
\end{verbatim}

\begin{Shaded}
\begin{Highlighting}[]
\KeywordTok{library}\NormalTok{(tidyr)}
\KeywordTok{library}\NormalTok{(ggplot2)}
\KeywordTok{library}\NormalTok{(caret) }\CommentTok{# need to be installed }
\end{Highlighting}
\end{Shaded}

\begin{verbatim}
## Loading required package: lattice
\end{verbatim}

\subsection{Load table\_final}\label{load-table_final}

\begin{Shaded}
\begin{Highlighting}[]
\KeywordTok{load}\NormalTok{(}\StringTok{"C:}\CharTok{\textbackslash{}\textbackslash{}}\StringTok{Users}\CharTok{\textbackslash{}\textbackslash{}}\StringTok{vercello}\CharTok{\textbackslash{}\textbackslash{}}\StringTok{Dropbox}\CharTok{\textbackslash{}\textbackslash{}}\StringTok{QUT}\CharTok{\textbackslash{}\textbackslash{}}\StringTok{STEM}\CharTok{\textbackslash{}\textbackslash{}}\StringTok{Data}\CharTok{\textbackslash{}\textbackslash{}}\StringTok{table_final.Rdata"}\NormalTok{)}
\end{Highlighting}
\end{Shaded}

\subsection{One more step!}\label{one-more-step}

Before estimating the accuracy, we are going to create a new variable
named ``Response''. Response is equal to 1 if your classification was
correct and 0 if not.

\begin{Shaded}
\begin{Highlighting}[]
\NormalTok{table_final<-table_final}\OperatorTok\KeywordTok{mutate}\NormalTok{(}\DataTypeTok{Response=} \KeywordTok{ifelse}\NormalTok{(}\StringTok{`}\DataTypeTok{Classification 1}\StringTok{`}\OperatorTok{==}\StringTok{`}\DataTypeTok{Classification 2}\StringTok{`}\NormalTok{,}\DecValTok{1}\NormalTok{,}\DecValTok{0}\NormalTok{))}
\end{Highlighting}
\end{Shaded}

\subsection{1. Overall Accuracy}\label{overall-accuracy}

The first method to estimate your accuracy is such that:

\[
\begin{aligned}
Accuracy=\frac{\text{number of correct classifications}}{\text{total number of classifications}}
\end{aligned}
\]

\begin{Shaded}
\begin{Highlighting}[]
\NormalTok{acc1<-table_final}\OperatorTok\KeywordTok{group_by}\NormalTok{(mediaID)}\OperatorTok\KeywordTok{summarize}\NormalTok{(}\DataTypeTok{Accuracy=}\KeywordTok{sum}\NormalTok{(Response)}\OperatorTok{/}\DecValTok{15}\NormalTok{)}
\end{Highlighting}
\end{Shaded}

It can happen that you have classified the same image more than one
time, if it's the case, Accuracy is greater than 1. Let's look at it.

\begin{Shaded}
\begin{Highlighting}[]
\NormalTok{check1<-table_final}\OperatorTok\KeywordTok{group_by}\NormalTok{(mediaID)}\OperatorTok\KeywordTok{tally}\NormalTok{()}\OperatorTok\KeywordTok{filter}\NormalTok{(n}\OperatorTok{>}\DecValTok{15}\NormalTok{)}
\NormalTok{check1}
\end{Highlighting}
\end{Shaded}

\begin{verbatim}
## # A tibble: 15 x 2
##        mediaID     n
##          <dbl> <int>
##  1 10001025802    30
##  2 12001091202    30
##  3 12025019701    30
##  4 12033068201    30
##  5 14009044302    30
##  6 14009047302    30
##  7 15020223501    30
##  8 35004144301    30
##  9 35023133801    30
## 10 35024241206    30
## 11 35025040702    30
## 12 35025042002    30
## 13 35025042202    30
## 14 35027250302    30
## 15 35029034602    30
\end{verbatim}

If check1 is empty - you haven't classified an image more than 1 time.
If not, the following lines need to be run

\begin{Shaded}
\begin{Highlighting}[]
\NormalTok{table_final<-table_final}\OperatorTok\KeywordTok{group_by}\NormalTok{(mediaID)}\OperatorTok\KeywordTok{slice}\NormalTok{(}\DecValTok{1}\OperatorTok{:}\DecValTok{15}\NormalTok{)}
\end{Highlighting}
\end{Shaded}

\begin{Shaded}
\begin{Highlighting}[]
\NormalTok{check2<-table_final}\OperatorTok\KeywordTok{group_by}\NormalTok{(mediaID)}\OperatorTok\KeywordTok{tally}\NormalTok{()}\OperatorTok\KeywordTok{filter}\NormalTok{(n}\OperatorTok{>}\DecValTok{15}\NormalTok{)}
\NormalTok{check2 }\CommentTok{# have to be 0}
\end{Highlighting}
\end{Shaded}

\begin{verbatim}
## # A tibble: 0 x 2
## # ... with 2 variables: mediaID <dbl>, n <int>
\end{verbatim}

Recompute your accuracy

\begin{Shaded}
\begin{Highlighting}[]
\NormalTok{acc1<-table_final}\OperatorTok\KeywordTok{group_by}\NormalTok{(mediaID)}\OperatorTok\KeywordTok{summarize}\NormalTok{(}\DataTypeTok{Accuracy=}\KeywordTok{sum}\NormalTok{(Response)}\OperatorTok{/}\DecValTok{15}\NormalTok{)}
\end{Highlighting}
\end{Shaded}

\begin{itemize}
\tightlist
\item
  your maximum accuracy was
\end{itemize}

\begin{Shaded}
\begin{Highlighting}[]
\KeywordTok{max}\NormalTok{(acc1}\OperatorTok{$}\NormalTok{Accuracy)}
\end{Highlighting}
\end{Shaded}

\begin{verbatim}
## [1] 1
\end{verbatim}

\begin{itemize}
\tightlist
\item
  your minimum accuracy was
\end{itemize}

\begin{Shaded}
\begin{Highlighting}[]
\KeywordTok{min}\NormalTok{(acc1}\OperatorTok{$}\NormalTok{Accuracy)}
\end{Highlighting}
\end{Shaded}

\begin{verbatim}
## [1] 0
\end{verbatim}

\begin{itemize}
\tightlist
\item
  your average accuracy was
\end{itemize}

\begin{Shaded}
\begin{Highlighting}[]
\KeywordTok{mean}\NormalTok{(acc1}\OperatorTok{$}\NormalTok{Accuracy)}
\end{Highlighting}
\end{Shaded}

\begin{verbatim}
## [1] 0.2780488
\end{verbatim}

\subsection{2. Accuracy for Hard Corals
only}\label{accuracy-for-hard-corals-only}

\begin{Shaded}
\begin{Highlighting}[]
\NormalTok{acc2<-table_final}\OperatorTok\KeywordTok{group_by}\NormalTok{(mediaID)}\OperatorTok\KeywordTok{filter}\NormalTok{(}\StringTok{`}\DataTypeTok{Classification 1}\StringTok{`}\OperatorTok{==}\StringTok{"Hard Corals"}\NormalTok{)}\OperatorTok
\StringTok{  }\KeywordTok{summarize}\NormalTok{(}\DataTypeTok{Accuracy_coral=}\KeywordTok{sum}\NormalTok{(Response)}\OperatorTok{/}\KeywordTok{n}\NormalTok{())}
\KeywordTok{max}\NormalTok{(acc2}\OperatorTok{$}\NormalTok{Accuracy_coral)}
\end{Highlighting}
\end{Shaded}

\begin{verbatim}
## [1] 1
\end{verbatim}

\begin{Shaded}
\begin{Highlighting}[]
\KeywordTok{min}\NormalTok{(acc2}\OperatorTok{$}\NormalTok{Accuracy_coral)}
\end{Highlighting}
\end{Shaded}

\begin{verbatim}
## [1] 1
\end{verbatim}

\begin{Shaded}
\begin{Highlighting}[]
\KeywordTok{mean}\NormalTok{(acc2}\OperatorTok{$}\NormalTok{Accuracy_coral)}
\end{Highlighting}
\end{Shaded}

\begin{verbatim}
## [1] 1
\end{verbatim}

\subsection{3. Another approach}\label{another-approach}

A second to estimate your accuracy follows this equation:

\[
\begin{aligned}
Accuracy=\frac{\text{TP+TN}}{\text{TP+TN+FP+FN}}
\end{aligned}
\] Where in our example: TP= true positives - you and the expert
classified hard corals; TN= true negatives - only the expert classified
hard corals, you chose something else; FP= false positives - you and the
expert didn't classified hard corals; FN= false negatives - only you
classified hard corals, the expert chose something else

Add TP, TN, FP and FN as variables in the table\_final such as they are
equal to 1 is the condition is true, 0 otherwise.

\begin{Shaded}
\begin{Highlighting}[]
\NormalTok{table_final<-table_final}\OperatorTok\KeywordTok{group_by}\NormalTok{(mediaID,id)}\OperatorTok\KeywordTok{filter}\NormalTok{(}\StringTok{`}\DataTypeTok{Classification 1}\StringTok{`}\OperatorTok{==}\StringTok{"Hard Corals"}\NormalTok{)}\OperatorTok
\StringTok{  }\KeywordTok{mutate}\NormalTok{(}\DataTypeTok{TP=}\KeywordTok{ifelse}\NormalTok{(}\StringTok{`}\DataTypeTok{Classification 1}\StringTok{`}\OperatorTok{==}\StringTok{"Hard Corals"} \OperatorTok{&}\StringTok{ `}\DataTypeTok{Classification 2}\StringTok{`}\OperatorTok{==}\StringTok{"Hard Corals"}\NormalTok{,}\DecValTok{1}\NormalTok{,}\DecValTok{0}\NormalTok{))}\OperatorTok
\StringTok{           }\KeywordTok{mutate}\NormalTok{(}\DataTypeTok{TN=}\KeywordTok{ifelse}\NormalTok{(}\StringTok{`}\DataTypeTok{Classification 1}\StringTok{`}\OperatorTok{!=}\StringTok{"Hard Corals"} \OperatorTok{&}\StringTok{ `}\DataTypeTok{Classification 2}\StringTok{`}\OperatorTok{!=}\StringTok{"Hard Corals"}\NormalTok{,}\DecValTok{1}\NormalTok{,}\DecValTok{0}\NormalTok{))}\OperatorTok
\StringTok{                    }\KeywordTok{mutate}\NormalTok{(}\DataTypeTok{FP=}\KeywordTok{ifelse}\NormalTok{(}\StringTok{`}\DataTypeTok{Classification 1}\StringTok{`}\OperatorTok{!=}\StringTok{"Hard Corals"} \OperatorTok{&}\StringTok{ `}\DataTypeTok{Classification 2}\StringTok{`}\OperatorTok{==}\StringTok{"Hard Corals"}\NormalTok{,}\DecValTok{1}\NormalTok{,}\DecValTok{0}\NormalTok{))}\OperatorTok\StringTok{  }
\StringTok{                             }\KeywordTok{mutate}\NormalTok{(}\DataTypeTok{FN=}\KeywordTok{ifelse}\NormalTok{(}\StringTok{`}\DataTypeTok{Classification 1}\StringTok{`}\OperatorTok{==}\StringTok{"Hard Corals"} \OperatorTok{&}\StringTok{ `}\DataTypeTok{Classification 2}\StringTok{`}\OperatorTok{!=}\StringTok{"Hard Corals"}\NormalTok{,}\DecValTok{1}\NormalTok{,}\DecValTok{0}\NormalTok{))}
\end{Highlighting}
\end{Shaded}

Calculate your accuracy using the equation

\begin{Shaded}
\begin{Highlighting}[]
\NormalTok{acc3<-table_final}\OperatorTok\KeywordTok{group_by}\NormalTok{(mediaID)}\OperatorTok\KeywordTok{summarize}\NormalTok{(}\DataTypeTok{Accuracy2=}\KeywordTok{sum}\NormalTok{(TP}\OperatorTok{+}\NormalTok{FP)}\OperatorTok{/}\KeywordTok{sum}\NormalTok{(TP}\OperatorTok{+}\NormalTok{TN}\OperatorTok{+}\NormalTok{FP}\OperatorTok{+}\NormalTok{FN))}
\end{Highlighting}
\end{Shaded}

\subsection{4. Another approach (again) - confusion
matrix}\label{another-approach-again---confusion-matrix}

\subsection{5. Questions}\label{questions}

\begin{enumerate}
\def\labelenumi{\arabic{enumi}.}
\tightlist
\item
  Caculate ``acc3'' using another the Sand category as condition,
  discuss about your new accuracy
\item
  Explain the outputs from the confusion matrix and statistics using the
  R documentation - {[}hint{]}
  \url{https://www.rdocumentation.org/packages/caret/versions/6.0-81/topics/confusionMatrix}
\item
  In overall, are you ready to be a coral expert?
\end{enumerate}


\end{document}
